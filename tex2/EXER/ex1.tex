\chapter{Off-lattice Monte Carlo simulations}


\section{The Hard Sphere model}

The \textit{Hard Sphere model} is a paradigmatic model in Soft Matter (despite the naming would not suggest so). The interaction energy between two hard spheres is zero if the separation distance is larger than the diameter of the particle $\sigma$ and infinite otherwise. Set $\sigma=1$ as the unit of length. Using the Hard Sphere model, one can simplify the Metropolis acceptance rule as follows:
\begin{itemize}
	\item reject every displacement that brings any two particles closer than $\sigma$; call it an overlap. You can assign a very large energy to the configuration if an overlap is present, thus overlap can be spotted more easily.
	\item accept every displacement that keeps every pair of particles at a distance larger than $\sigma$ or that removes an overlap.
\end{itemize}
An acceptable state for a Hard Sphere system has zero energy and is homogeneously distributed in the simulation box. Consider a system of $N = 200$ Hard Spheres in three dimensions. Starting from random conformations, simulate the system at different number density $\rho = \frac{N}{V}: (i)\> \rho = 0.05\sigma^{-3} (ii)\> \rho = 0.4\sigma^{-3}$ and $(iii)\> \rho = 0.62\sigma^{-3}$. Use a simple cubic box and pay attention to set a suitable maximum displacement. Check that
the energy reaches zero (equilibration) before collecting data. Compute the radial distribution function
$g_2(r)$, discussed in class and describe what changes between the three different states.
\textbf{\\ Resolution \\}
The model we are implementing is based on the hard-sphere nature of the constituent molecules. Formally, we can write the Hamiltonian of the system as:
$$
\mathcal{H} = T + V = \sum_i \frac{1}{2}mv_i^2 + U
$$
where $T$ represents the standard kinetic energy and $U$ is the total potential energy of the system, composed of pair-wise interactions $U = \sum_{i\neq j}U_{ij}$. Specifically, the hard-sphere condition is realized by imposing\footnote{Note that we have rescaled the spatial lengths by choosing $\sigma = 1$, where $\sigma$ is the molecular diameter. In this report, all lengths will therefore be dimensionless}:
\begin{equation}
	U_{ij}=
	\begin{cases}
		\infty \>\> \text{       if       } \>\> |\vec r_i - \vec r_j | < 1 \\
		0 \>\> \text{        if       } \>\>|\vec r_i - \vec r_j | > 1 
	\end{cases}
\end{equation}
Thus, the molecules interact only to avoid spatial overlaps but are otherwise free to move without experiencing any force. In the NVT ensemble (the one sampled with Metropolis), the probability distribution function in phase space $(\vec r_i, \vec p_i)$ is given by the Boltzmann formula:
$$
p(\{\vec{r},\vec{p}\}) = \frac{1}{Z}\exp(-\beta \mathcal{H}(\{\vec r, \vec p\}))
$$ 
However, in this specific context, we are interested only in the marginalized configurational probability distribution, i.e., the probability of observing a specific configuration in coordinate space:
$$
p(\{\vec{r}\}) = \int d\vec{p}_1d\vec{p}_2 \dots d\vec p_n \> p(\{\vec{r},\vec{p}\}) 
$$
This expression factorizes conveniently due to the fact that the potential energy depends only on the configuration $\{\vec r_i\}$ and the kinetic energy only on the momenta $\{\vec p_i\}$:
$$
p(\{\vec{r}\}) = p(\vec{r}_1, \vec{r}_2, \dots \vec{r}_n) = \frac{1}{Z_{conf}}\exp(-\beta U) = \frac{1}{Z_{conf}}\exp\left(-\beta \sum_{i\neq j}U_{ij}\right)
$$
This is the probability distribution we want to sample from. In the hard-sphere model, this further simplifies to:
\begin{equation}
	p(\{\vec{r}\}) =
	\begin{cases}
		const \neq 0  \>\>\> \text{ if no overlap occurs} \\
		0  \>\>\> \text{ if even one overlap occurs} \\
	\end{cases}
	\label{eq1}
\end{equation}
This implies that all \textit{acceptable} states (i.e. configurations with no overlaps) are equiprobable. The Metropolis algorithm that will be implemented in the next section is designed to generate samples drawn precisely from this probability distribution. Before proceeding with the implementation of the algorithm, a few remarks:
\begin{itemize}
	\item The derivation of the marginalized configurational probability was carried out in the canonical ensemble (NVT). However, due to the specific form of the potential, there is effectively no distinction in this context between the microcanonical (NVE) and the canonical ensembles. Indeed, the dependence on the thermal parameter $\beta$ vanishes because the potential is either zero or infinite, rendering the Boltzmann factor independent of temperature for all allowed configurations.
	\item We construct Markov chains designed to sample from the PDF in Eq.\ref{eq1}. It is important to note that this type of MC simulation is inherently static, meaning that we are sampling only the configurational component of the full distribution. Consequently, our MCMC algorithm yields configurations sampled with the correct probability, but no information regarding velocities can be inferred from these purely spatial samples. We can, therefore, compute only static or structural properties (such as the radial distribution function) that depend solely on the configurational subspace.
\end{itemize}

\subsection{The algorithm}
This section briefly summarizes the implementation of the algorithm in C++. A base structure \texttt{Particle} was defined to store a $d$-dimensional tuple of coordinates. The core class of the codebase is structured as follows:

\begin{lstlisting}[language=C++]
	template<int Dim>
	class NDMolDyn {
		std::vector<Particle> particles;
		std::vector<std::vector<int>> verletLists;
		std::vector<int> headOfChain;
		std::vector<Particle> backupPositions;
	};
\end{lstlisting}
where:\begin{itemize}
	\item The vector \texttt{particles} contains all $N$ constituent particles, where $N$ is defined by the input parameters.
	\item The matrix \texttt{verletLists} implements the \textit{Verlet list strategy}. Specifically, the element \texttt{verletLists[i]} is a list containing the indices of particles that are sufficiently close to the $i$-th particle. Constructing these lists is an effective method to accelerate the force calculation. In fact, when checking if a particle $i$ overlaps with any other particles, we only need to iterate through its Verlet list, whose size is typically much smaller than $N$. This eliminates the need to check all possible particle pairs.
	\item The process of generating the Verlet lists itself can be computationally intensive, potentially introducing a new $O(N^2)$ bottleneck. For this reason, the vector \texttt{headOfChain} implements the \textit{cell list strategy}, which converts the costly Verlet list construction from an $O(N^2)$ operation to a highly efficient $O(N)$ algorithm. The vector \texttt{backupPositions} is used to evaluate whether it is necessary to rebuild the Verlet lists (rebuilding is avoided at each iteration thanks to the presence of a safety parameter, \texttt{skin}). The linked-list structure needed for the cell list approach is not contained in a separate vector $L$, but is included directly into the \texttt{Particle} structure for implementation simplicity:
	\begin{lstlisting}[language=C++]
	template<int Dim>
	struct Particle{
		std::array<double, Dim> position;
		int next = -1;    // -1 means nullptr
	};
	\end{lstlisting}
\end{itemize}
The MC algorithm goes as follows\footnote{This is just a minimal example, the full code is longer because there are additional feature (acceptance rate counter, ...)}:
\begin{lstlisting}[language=C++]
#define vecd std::array<double, Dim>

//Run a single step of MC (thus N proposals)
template<int Dim>
void NDMolDyn<Dim>::singleMCStep(){
	// First of all, select a random particle
	int particle_idx = rand() % m_particles.size();
	vecd old_pos = m_particles[particle_idx].position;
	vecd new_pos;
	for(int k = 0; k < Dim; ++k){
		double delta = (static_cast<double>(rand()) / RAND_MAX) * 2.0 * max_displacement - max_displacement;
		new_pos[k] = old_pos[k] + delta;
		if(new_pos[k] >= m_L){ //PBC
			new_pos[k] -= m_L;
		} else if(new_pos[k] < 0.0){
			new_pos[k] += m_L;
		}
	}
	// Now that we have the new_pos, verify whether there is an overlap
	for(auto it = m_verletLists[particle_idx].begin(); it != m_verletLists[particle_idx].end(); ++it){
		vecd pos_j = m_particles[*it].position;
		double d2 = dist_sq_pbc(new_pos, pos_j);
		if(d2 < 1.0){ //Overlap detected! Return and don't do anything
			return;
		}
	}
	// If we reach this point, we can accept the move
	m_particles[particle_idx].position = new_pos;
	// Finally, check whether we need to rebuild the verlet lists
	double trigger = distance_square_pbc(m_backupParticles[particle_idx], new_pos);
	if (trigger > m_skin/2 * m_skin/2) {
		// Rebuild the verlet lists
		buildVerletLists();
	}
}
\end{lstlisting}

The simulation can be run in an arbitrary number of dimension, but we'll use $d = 2$ or $d=3$ for the sake of simplicity. The parameters that we can modify can be summarized:


\begin{table}[h]
	\centering
	\begin{tabular}{ll}
		$\rho$ & Density \\
		$N$    & Number of particles   \\
		$r_{cut}$ & Cut-off radius, Verlet's technique    \\
		$r_{skin}$ & Skin, Verlet's technique \\
		$M$ & Number of cell per dimension \\
		$MCSteps$ & Number of MC steps to perform (1 MCStep = $N$ proposals)      
	\end{tabular}
\end{table}


When the function \texttt{buildVerletLists()} is called, the program will start filling the object \texttt{std::vector<std::vector<int>> verletLists}. In particular, the list \texttt{verletLists[i]} will consist of all particles whose position $r_j$ is such that:
$$
|\vec r_j - \vec r_i| < r_{cut} + r_{skin}
$$
The parameter $r_{cut}$ should be chosen as to represent a characteristic length over which the interaction dies out. In this specific case, since the pair-wise hard-sphere interaction is non zero only when $|\delta r| < 1$ (diameter of the molecules), we can safely set $r_{cut}=1$. The skin parameter is a performance parameter, meaning that it sets how frequently one has to update the Verlet lists (the less, the better). In this code, I've used to rule according to which when the total displacement of a molecules (with respect to the position occupied the last time \texttt{buildVerletList()} was called) is larger than half of the skin parameter, then the Verlet lists must be rebuilt from scratch. A good choice is $r_{skin} = 0.5$ (but this depends on the specific run). In order for the algorithm to work, a few sanity check are in order:
\begin{itemize}
	\item The Verlet lists are built upon the cell lists, meaning that the neighbours of a particle are selected by only checking adjacent cells. If the linear size of the cells is too small, then we might be losing essential interactions. Specifically, we require:
	\begin{equation}
		M < \Bigl\lfloor \frac{L}{r_{cut} + r_{skin}} \Bigr\rfloor
	\end{equation}
	\item At each MC step, $N$ particles are randomly selected and a spatial move is proposed according to a parameter \texttt{max\_displacement}. In order for this move not to break the Verlet list inner working, we request:
	\begin{equation}
		\text{max\_displacement} < \frac{1}{2} d_{skin}
	\end{equation}
\end{itemize}
The program will automatically perform those sanity check at the beginning of the simulation and signal to the user whether a violation has been detected. 

\subsection{Qualitative analyis}
We set $N = 200$ particles and investigate the behavior of the system in $3D$ across three different number densities: $\rho = 0.62, \rho = 0.4$, and $\rho = 0.05$. The simulation executes the MCMC algorithm and records the coordinates of all particles at specified intervals, dumping them into a \texttt{traj.xyz} file. This output is particularly useful when analyzed with \texttt{OVITO}, a free software package capable of reading \texttt{.xyz} trajectory files and rendering the simulation in $3D$.\footnote{The native $2D$ rendering option was implemented using the SFML library. Instead, due to the complexity of real-time $3D$ rendering, I relied on an external third-party software.} The resulting configurations for the three defined densities are visualized in Fig.\ref{fig:OVITO}: the results qualitatively confirm that higher concentration leads to denser molecular packing. Table.\ref{table:table1} presents the parameters defining the MC chain along with key diagnostic quantities. These diagnostics include the acceptance rate and the rebuild probability (which quantifies the frequency of the computationally expensive Verlet list updates).
\begin{table}
	\centering
	\begin{tabular}{|c|c|c|c|c|c|}
		\hline
		$\rho$ &  Max Displacement & Skin & M & Rebuilding & Acceptance rate\\
		\hline
		$0.05$ &  0.4 & 1 & 5 &14 \%&$0.92 \pm 0.02$ \\

		$0.4$  &  0.35 & 0.8 & 4 & 10 \%&$0.48 \pm 0.05$ \\
		
		$0.62$  &  0.2 & 0.5 & 4&4\%&$0.38 \pm 0.06$ \\ 
		\hline
	\end{tabular}
	\caption{\textit{A list of the parameters governing the Verlet and cell list techniques that were implemented to enhance code performance ($\texttt{max\_displacement, skin, M}$). The total number of MC steps and $r_{cut}$ were held constant at $10^5$ and $1$, respectively. In addition, two diagnostic parameters were monitored to ensure the Markov chain explored the space correctly and that the acceleration technique was effective: the acceptance rate and the rebuilding factor. The acceptance rates for the higher densities ($\rho = 0.4$ and $\rho = 0.62$) can be considered satisfactory, whereas the rate observed for $\rho = 0.05$ is rather high. While in ordinary MC simulations this might suggest the algorithm is exploring the space too slowly, this is not the case here: the low number density implies extremely rare interactions, hence a higher acceptance rate.}}
	\label{table:table1}
\end{table}
\begin{figure}
	\centering
	\includegraphics[scale=0.33]{./FIG/gas.png}
	\includegraphics[scale=0.33]{./FIG/liquid.png}
	\includegraphics[scale=0.33]{./FIG/solid.png}
	\caption{\textit{3D rendering (under periodic boundary condition) performed by OVITO starting from three MC chains. From the left to the right, $\rho = 0.05, \rho = 0.4, \rho = 0.62$ and $N$ is kept constant at $N=200$. The spatial lengths are all rescaled in such a way that $\sigma=1$, where $\sigma$ is the molecules diameter.}}
	\label{fig:OVITO}
\end{figure} 
With the obtained MC chains for the spatial configurations, we can analyze the structural properties of the system, starting with the \textit{radial distribution function} $g(r)$. For this purpose, \texttt{OVITO} was used to calculate the instantaneous, frame-by-frame radial distributions. These instantaneous distributions were subsequently averaged (the Python notebook) to yield a smoother, statistically reliable curve. The resulting time-averaged $g(r)$ curves are shown in the right panel of Fig.\ref{fig:g(r)}. For all simulated densities, $g(r)$ vanishes when $r < 1$. This confirms the expected behavior and serves as a successful sanity check for the algorithm. As spatial overlap is strictly prohibited, no pair of particles can stay closer than the molecular diameter ($r_{\star} = 1$), and $g(r)$ is precisely zero within this exclusion zone. 

The behaviour for $r > 1$ suggests the formation of at least two phases (a gaseous phase and a condensed one). When $\rho = 0.05$, $g(r)$ remains close to the asymptotic line $g(r)=1$, which is characteristic of an ideal gas. In other words, at sufficiently low densities, the local density is relatively homogeneous and comparable to the global density $\rho = N/V$. A slight enhancement is observed at $r \approx 1$, which decays back to $1$ as $r\to \infty$; this indicates that the local density in the immediate vicinity of a molecule is slightly higher than the typical (uncorrelated) density of an ideal gas. This effect is a direct consequence of the excluded volume interaction. Consider a particle fixed at $r=0$: in an ideal gas system, a second particle can arbitrarily occupy any point $\vec r$ in space independent of the existence of the first particle at $\vec r = 0$. Consequently, there would be a finite probability $p(V_\sigma)$ of finding the second particle within the region $V_\sigma$, defined as a sphere of radius $\sigma$ centered at the origin. Getting back to a system with excluded volume interactions, this probability $p(V_\sigma)$ must necessarily be zero, since the potential energy of an overlapping configuration is infinite. Consequently, this probability mass $p(V_\sigma)$ is effectively ''redistributed'' into the accessible region of space immediately surrounding the sphere $V_\sigma$.

When $\rho = 0.4$ or $\rho = 0.62$, the radial distribution curves exhibit more complex features. The peak around $r=1$ becomes significantly more pronounced, increasing in height as $\rho$ increases. Furthermore, the curve displays the characteristic behavior of oscillations exponentially damped that asymptotically decay to the value $1$ as $r \to \infty$. This pattern provides us with crucial information regarding the geometric structure of the system. The alternating peaks and valleys correspond to successive \textit{coordination shells}. The first peak represents the nearest neighbors forming a "cage" around the reference particle, while the subsequent peaks represent layers of secondary neighbors. The first valley implies a region of particle depletion: this is expected, as the presence of a tightly packed first shell prohibits other particles from occupying the same volume due to local exclusion. The damping of these oscillations indicates that the positional correlation is eventually lost at long distances.

Clearly enough, the larger the density $\rho$, the more pronounced those oscillations are (since the geometric packing become more and more regular).

\begin{figure}
	\centering
	\includegraphics[width=0.49\linewidth]{./FIG/RDF_plot.pdf}
	\includegraphics[width=0.49\linewidth]{./FIG/RDF_plot_well.pdf}
	\caption{\textit{(Left panel) Radial distribution function for the hard-sphere model as a function of (rescaled) distance $r$ at different values of $\rho$ (but $N = 200$). In all cases, $g(r)=0$ when $r < 1$, thanks to the excluded volume, and the correlation dies out when $r$ gets larger  (Right panel) Radial distribution function for the Van der Waals model (isotropic patchy hard sphere where $\theta_{max} = \frac{\pi}{2}$). Now a clear correlation appears for all three densities in the region $ 1 < r < 1.2$.}}
	\label{fig:g(r)}
\end{figure}
\section{The Patchy Hard Sphere model}
\subsection{Implementation}
To implement the patchy interaction described by the Kern-Frenkel potential, the \texttt{Particle} structure was extended to include a \texttt{Patch} object, which essentially represents a three-dimensional unit vector. Although each particle is decorated with two patches, their diametrically opposite arrangement allows for a significant simplification: for practical purposes, tracking a single unit vector is sufficient to define the orientation. To evaluate the pairwise interaction energy, the function \texttt{getEnergy()} was developed:
\begin{lstlisting}[language=C++]
double NDMolDyn<3>::getEnergy(vecd particle_pos, Patch particle_patch, int particle_idx) const {
	double energy = 0.0;
	for(auto it = m_verletLists[particle_idx].begin(); it != m_verletLists[particle_idx].end(); ++it){
		vecd pos_j = m_particles[*it].position;
		double d2 = dist_sq_pbc(particle_pos, pos_j);
		// If within interaction range (we've already checked for overlaps)
		if (d2 < (1.0 + 0.2)*(1.0 + 0.2)) {
			double cos_angle_1 = computeAngle(particle_patch, particle_pos, pos_j, d2);
			double cos_angle_2 = computeAngle(m_particles[*it].getPatch(), pos_j, particle_pos, d2);
			if (std::abs(cos_angle_1) > m_cosThetaMax && std::abs(cos_angle_2) >  m_cosThetaMax) { 
				energy += -1.0; //Dimensionless energy, assume epsilon_SW = 1
			}
		}   
	}
	return energy;
}
\end{lstlisting}
\subsection{Extreme cases}
Before considering a reasonable set of values for $\theta_{max}$, let's explore the extreme cases, where $\theta_{max} = 0$ and $\theta_{max} = \frac{\pi}{2}$. In the first case, the molecules needs to be completely and perfectly aligned to interact; this is impossible (because of machine precision), so when $\theta_{max} = 0$ we fallback to the previous case where no interactions (apart from the repulsive volume) occurs. However, when $\theta_{max} = \frac{\pi}{2}$ then particles will always interact as long as their distance is smaller than $1+\delta = 1+0.2$ (but larger than 1). This is the case where no patchy-like interactions occur and only a spherically symmetric potential well around the molecules is considered (a sort of Van der Walls gas!)

We fix $\theta_{max} = \frac{\pi}{2}$ and evaluate the radial distribution function at $k_B T = 0.8$ for three different densities\footnote{Energy is rendered dimensionless by normalizing to the potential well depth, that is by setting $\epsilon_{SW} = -1$.}. The results are presented in the right panel of Fig.\ref{fig:g(r)}. Once again, for $r < 1$, the radial distribution vanishes due to the excluded volume interaction. However, particles are now subject to a short-range attractive potential well (energy gain of $-1$) which favors configurations where particles are directly surrounded by other particles. Indeed, Fig.\ref{fig:g(r)} confirms the presence of a region $1 < r < 1.2$ where the local density significantly exceeds the bulk density, suggesting the formation of small molecular clusters. Beyond the interaction range ($r > 1.2$), we recover the typical oscillatory behavior for high densities or the flat asymptotic profile for low densities. 

In Fig.\ref{fig:OVITOisotropic}, we present three simulation snapshots rendered with OVITO, corresponding to the configurations analyzed in the right panel of Fig.\ref{fig:g(r)}, where an isotropic potential well is included. Particles separated by a distance less than $1.2$ are visually connected by a bond, representing a direct (negative) contribution to the potential energy. The case of $\rho = 0.05$ is particularly interesting: although the system remains in a gaseous phase, the presence of a short-range attractive potential renders small molecular clusters thermodynamically stable (and energetically favorable). Consequently, we observe the formation of pairs (dimers), and with rapidly decreasing probability, triplets (trimers) or higher-order structures.

One might argue that the formation of these structures is merely coincidental, potentially occurring even in the absence of a potential well (these apparent bonds could simply result from random spatial fluctuations where particles momentarily approach each other at a distance smaller than $1.2$)\footnote{Bonds in the case where no potential well exists doesn't make much sense, since they do not contribute to the energy of the system. The point I'm trying to prove here is that, when the potential is switched on, the fact that a certain amount of molecules tend to cluster together in small structure cannot be simply explained by pure luck, i.e. particles that eventually come closer.}. To verify that the emergence of these structures is effectively driven by the short-range interaction, we quantify the number of bonds in both cases (i.e., $\epsilon = 0$ and $\epsilon = -1$), defining a bond between two particles whenever their mutual distance is less than $1.2$.

I've run a simulation with $10k$ MC steps under those conditions and averaged the number of bonds $b$ for the last $1000$ steps. When $\epsilon = 0$, we measure $b = 16 \pm 4$, whereas when $\epsilon = -1$ we obtain $b = 52 \pm 5$. This is enough evidence to prove that the short range potential is responsible for the formation of those very small structures (only a few particles wide).

\begin{figure}
	\centering
	\hspace{-0.5cm}
	\includegraphics[scale=0.33]{./FIG/gas_well1.png}
	\hspace{0.1cm}
	\includegraphics[scale=0.33]{./FIG/liquid_well1.png}
	\hspace{0.1cm}
	\includegraphics[scale=0.33]{./FIG/solid_well1.png}
	\hspace{-0.5cm}
	\caption{\textit{3D rendering (under periodic boundary condition) performed by OVITO starting from three MC chains (isotropic potential well + excluded volume). From the left to the right, $\rho = 0.05, \rho = 0.4, \rho = 0.62$ and $N$ is kept constant at $N=200$. The blue bonds connect pair of molecule whose distance is less than $1.2$ (thus interacting particles).}}
	\label{fig:OVITOisotropic}
\end{figure} 

\paragraph{Energy and condensation}
An alternative approach to investigating bonds formation involves the analysis of the system's potential energy. Indeed, a direct relationship exists such that $U = -Nb$, where $b$ denotes the number of bonds. The evolution of the potential energy as a function of MC steps is illustrated in Fig.\ref{fig:3traceplots}, alongside a histogram of the same observable at a temperature of $k_B T = 0.8$. As expected, the MCMC algorithm begins sampling configurations from the correct equilibrium distribution only after a transient period, which depends on the chosen initial state. Following this initial \textbf{burn-in} phase, the trace plots clearly demonstrate that the Monte Carlo chains have converged to the limiting distribution and are sampling effectively. The distribution of the samples is approximately Gaussian, a result consistent with standard expectations in statistical mechanics that serves as a validity check. To obtain a Monte Carlo estimate of the thermal average $\langle U \rangle$, we discard the initial samples (burn-in) and compute the mean over the remaining trajectory. Next, we can compute the average energy $\langle U \rangle$ for different values of $\rho$ and $T$ (Fig.\ref{fig:transphase}). 

\begin{figure}
	\centering
	\includegraphics[scale = .8]{./FIG/Energy_analysis_well.pdf}
	\caption{\textit{Trace plots (on the left) and relative histograms (on the right) for three MC chains sampling from the isotropic potential well system ($\cos(\theta_{max}) = \pi /2$) at fixed temperature $k_B T = 0.8$ for three different values of density $\rho$. Number of particles $N = 200$ and $10000$ MC steps. Both the trace plots and the histograms display the usual behaviour expected from a Markov chain-based algorithm (an initial transient + a thermalized phase).}}
	\label{fig:3traceplots}
\end{figure}

At sufficiently high temperatures (specifically, where ``high'' means much larger than the depth of the potential well), the effect of the short-range potential becomes essentially negligible, as particles have sufficient thermal energy to freely enter and exit the potential well. Consequently, these high-temperature systems effectively behave as systems governed purely by excluded volume interactions. The energy curve should thus relax to an asymptotic value $U_{free}$ that can be interpreted as the energy that the system would have if the probability distribution function on the configuration space was the one associated to the same system with no potential well. When the density is low enough ($\rho = 0.05$), then particles are usually far away from each other and we expect: 
$$
U_{free} = \lim_{T\to\infty} U_{well}(T) = 0
$$
However, when the density grows, the asymptotic energy $U_{free}$ becomes increasingly negative. This is simply due to the fact that, when the system is high-density, there will be a moderate probability that a pair of particles gets closer than $1.2$, thus creating a bond and adding a quanta $\epsilon=-1$ to the energetic budget. However, this does not imply that particles are actively driven together by the short-range attraction, as the energetic influence of this potential is essentially negligible when the temperature is sufficiently high. Instead, the observed asymptotic energy reflects the fact that particles are forced into close proximity purely by geometric packing constraints (excluded volume), rather than being directed by attractive forces.

\begin{figure}
	\centering
	\includegraphics[width=0.48\linewidth]{./FIG/average_energy_vs_well_depth.pdf}
	\includegraphics[width=0.48\linewidth]{./FIG/heat_capacity_vs_well_depth.pdf}
	\caption{\textit{On the left, curves representing the average potential energy $\langle U \rangle$ as a function of the temperature for various densities. Below a critical value $T_c$, it's thermodynamically more favorable for the system to condense and create dense pockets of particles (droplets) which minimizes the energy. This will, of course, decrease the entropy as well. But for low temperature, it's the energetic term that really contributes to the free energy. On the right, heat capacity plot (obtained by dividing the discrete $\Delta E$ by the interval $\Delta T$). For all densities, it seems like there is a peak around a critical value $T_c(\rho)$, a behavior reminiscent of phase transition}}
	\label{fig:transphase}
\end{figure}

An interesting behavior can be observed for low temperatures: once $T$ reaches a certain critical value, the average potential energy abruptly drops in a way that closely resembles a phase transition (enhanced by the peak of the heat capacity $\frac{\Delta E}{\Delta T}$). The low-temperature system is dominated by configurations with relatively high potential energy (in absolute value), meaning spatial configurations with lots of inter-particles bonds (since energy and bonds are connected). Put in other words, the gaseous phase has condensed and has formed droplets of dense liquid (Fig.\ref{fig:condensed}).

\begin{figure}
	\centering
	\includegraphics[scale = 0.5]{./FIG/cold_gas.png}
	\includegraphics[scale = 0.4]{./FIG/RDF_plot_well_gas_condensed.pdf}
	\caption{\textit{On the right, a snapshot from OVITO rendering for a potential-well system when $\rho = 0.05$ and $k_B T = 0.2$. What was before a gas of molecules randomly interacting becomes now a system where particles condense and create highly dense clusters with a large amount of internal bonds (thus decreasing the energy). On the left, the radial distribution function for the same configuration}}
	\label{fig:condensed}
\end{figure}


\subsection{Anisotropic patchy hard spheres}
Let us now consider a case where $\theta_{max} < \frac{\pi}{2}$ (anisotropy). When setting $\rho = 0.05$ and $k_B T = 0.8$, the system does not display any interesting behavior (other than the entropy-rich gaseous phase). Conversely, we expect that by decreasing the solid angle where the interaction potential is active, the probability of two particles successfully interacting is significantly reduced compared to the isotropic case ($\theta_{max} = \frac{\pi}{2}$). This reduction in effective attraction is analogous to lowering the critical freezing temperature of the system. Indeed, Fig.\ref{fig:thetamax} confirms this intuition: for $\theta_{max} = 60^\circ$ or $20 ^\circ$, a temperature of $k_B T = 0.8$ is too hot to observe any condensation phenomena.
\begin{figure}
	\centering
	\includegraphics[width = 0.48\linewidth]{./FIG/average_energy_vs_temp_cosThetamax.pdf}
	\includegraphics[width = 0.48\linewidth]{./FIG/average_energy_vs_temp_cosThetamax_zoom.pdf}
	\caption{\textit{(Left) Average potential energy vs temperature at fixed density $\rho = 0.05$ for the patchy hard sphere model for different values of $\theta_{max}$. When $\theta_{max} = 90^\circ$, then we essentially fallback to the isotropic case discussed in the previous paragraph. As $\theta_{max}$ approaches $0$, the system effectively behaves as if no potential well exists and the energy curve flattens out to $0$. The freezing point (critical $T_c$) becomes closer and closer to $0$ as $\theta_{max}$ approaches $0$. (Right) Focus on the region of very low temperature }}
	\label{fig:thetamax}
\end{figure}
Hence, let us simulate a system where $\rho = 0.05$, $k_B T = 0.01$, $\theta_{max} = \frac{\pi}{3}$ or $\theta_{max} = 0.35$ (for $100$k MC steps). Snapshots from OVITO rendering are plotted in Fig.\ref{fig:ovitopatch}
\begin{figure}
	\centering
	\includegraphics[width = 0.48\linewidth]{./FIG/linear_broad.png}
	\includegraphics[width = 0.48\linewidth]{./FIG/linear_narrow.png}
	
	\caption{\textit{Snapshots from the 3D OVITO render. Parameters of the patchy hard-spheres model: $\rho = 0.05$, $k_B T = 0.01$, $\theta_{max} = \frac{\pi}{6} \text{ (right) }, \theta_{max} = 0.35 \text{ (left) }$ after about $800k$ MC steps. }}
	\label{fig:ovitopatch}
\end{figure}
At this low temperature, the system tends to condense and form dense clusters of aligned particles. However, the structure of these clusters is now substantially different: previously, when the potential well was isotropic, the clusters were approximately spherical; now, however, they become increasingly \textbf{anisotropic and elongated} as the interaction cone narrows. Setting $\theta_{max} = 0.35$, for example, ensures the steric constraint such that the interaction cone of the potential is tight enough to accommodate only one particle per patch. Consequently, Fig.\ref{fig:thetamax} clearly demonstrates the formation of long linear chains of particles. When $\theta_{max} = \frac{\pi}{3}$, the cone is large enough to support at least $3$ particles, meaning that the chains are much wider and flexible. 

\paragraph{Bonds and energy}
With the help of OVITO, let's evaluate the average number of bonds in the system per each one of the run in Fig.\ref{fig:ovitopatch}. In particular, labeling as "wide" the case where $\theta_{max} = \frac{\pi}{3}$ and as "narrow" when $\theta_{max} = 0.35$:
\begin{equation}
	\begin{aligned}
		b_{wid} = 470 \pm 1 \\
		b_{nar} = 200 \pm 1
	\end{aligned}
\end{equation}
In Fig.\ref{fig:energyPatches} we illustrate the energy of the system as a function of MC steps (it took a lot of MC steps to reach thermal equilibrium and I was not even completely there yet (an usual problem when using Metropolis at low temperatures). However, $10^6$ MC steps implies approximately $100$ GB worth of data to be stored in the memory, so I couldn't proceed any further.). The average energy per particle $u$, taken considering only the last $10^5$ MC iteration is:
\begin{figure}
	\centering
	\includegraphics[scale = .8]{./FIG/Energy_trace_plots_patch.pdf}
	\caption{\textit{Energy over number of MC steps (up to $10^6$ steps) for a patchy-spheres model with $\theta_{max} = \pi / 3$ (left) , $\theta_{max} = 0.35$ (right). The thermalization process is not perfect, but this is an standard pathological behavior of Metropolis algorithm when $T$ is low.}}
	\label{fig:energyPatches}
\end{figure}

\begin{equation}
	\begin{aligned}
		u_{wid} &= -0.955 \pm 0.001 \\
		u_{nar} &= -2.272 \pm 0.002
	\end{aligned}
\end{equation}
Note that here it is not strictly true that $u = \frac{b}{N}$, since some OVITO counts as bonds all pair of particles whose distance is less then $1.2$, even though they might not be orientated in such a way that they really interact. Still, these observations allow us to extrapolate interesting properties based on the structural measure.

The average number of bonds per particle is approximately $\approx 4.7$ (wide patch) or $\approx 2$ (narrow patch), which is compatible with the chain structures observed in Fig.\ref{fig:ovitopatch}. In fact, when $\theta_{max} = 0.35$, almost all particles possess one bond per patch (thus two bonds in total), confirming the formation of long linear chains. Conversely, when $\theta_{max}=\frac{\pi}{3}$, each particle has on average $4.7$ bonds, equating to $2.35$ bonds per patch. This is still acceptable, since the geometric maximum for such an angle $\theta_{max}$ allows for at most $3$ bonds per patch. It is possible that, were the system allowed to thermalize further (by increasing the number of MC steps), we would observe a more regular structure with a constant number of bonds per particle.